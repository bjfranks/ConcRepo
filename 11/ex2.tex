% !TeX root = root.tex
% !TeX spellcheck = en_US
\section{On the Generality of our TSO Formalization}

\subsection{Task 1}
Let $\mathds{P}$ be the set of processes with $|\mathds{P}| = n$ and $\prod_{p \in \mathds{P}} (Q_p,\Sigma_p,\delta_p,q_{0_p})$ be the different programs. W.l.o.g. let $\mathds{A}^\prime = \mathds{A} \dot\cup \{x_i \mid 1 \leq i \leq n\}$ be a disjoint union and let $\mathds{D}^\prime = \mathds{D} \cup \{1\}$.

We construct the program $(Q,\Sigma,\delta,q_0)$ that encodes the different programs from above as follows:

We set $\Sigma$ to the set of all Operations over $\mathds{A}^\prime$ and $\mathds{D}^\prime$.

W.l.o.g. $Q = \dot\bigcup_{p \in \mathds{P}} Q_p \dot\cup \{q_0\}$ is a disjoint union.

The transition relation $\delta$ is the union of the relations of the original programs and the following additional transitions:
\begin{figure}[h]
\begin{tikzpicture}[shorten >=1pt,node distance=2cm,on grid,auto] 
   \node[state,initial] (q_0)   {$q_0$}; 
   \node (q_1) [right =of q_0] {$\vdots$};  
   \node[state] (q_01) [above =of q_1] {$q_{0_1}$};  
   \node[state] (q_0n) [below =of q_1] {$q_{0_n}$};
    \path[->] 
    (q_0) edge  node {arw($x_1$,0,1)} (q_01)
    (q_0) edge  node [below left] {arw($x_n$,0,1)} (q_0n);
\end{tikzpicture}
\end{figure}

This means that each process decides non-deterministically to execute an original program and sets the corresponding variable to 1 (which is initially set to 0). Then it executes the chosen original program. Since the atomic-read-write checks the variable for 0, it is not possible that two processes execute the same original program. Moreover, it is always possible for a process to choose a remaining program. Therefore, the control state reachability is preserved.

\subsection{Task 2}
Let $(Q,\Sigma,\delta,q_0)$ be a program and w.l.o.g. let $\mathds{A} = \{a_i \mid 1 \leq i \leq n\}$ for $n \in \mathds{N}$. Let $m_{\text{init}} : \mathds{A} \rightarrow \mathds{D}$ be a memory configuration.

We construct the program $(Q^\prime,\Sigma^\prime,\delta^\prime,q_0^\prime)$ with default memory configuration as follows:

We set $\mathds{A}^\prime = \mathds{A} \dot\cup \{\text{init},\text{ready}\}$ and $\mathds{D}^\prime = \mathds{D} \cup \{0,1\}$ where we assume w.l.o.g. that the addresses init and ready are not already in $\mathds{A}$. Then we define $\Sigma^\prime$ to be the set of all Operations over $\mathds{A}^\prime$ and $\mathds{D}^\prime$.

W.l.o.g. $Q^\prime = Q \dot\cup \bigcup_{i=0}^{n+1} w_i \dot\cup \{q_0^\prime\}$ is a disjoint union.

The transition relation $\delta^\prime$ is the same as $\delta$ extended by the following transitions:
\begin{figure}[h]
\begin{tikzpicture}[shorten >=1pt,node distance=5cm,on grid,auto] 
   \node[state,initial] (q_0)   {$q_0^\prime$}; 
   \node[state] (w_0) [right=of q_0] {$w_0$};
   \node[state] (w_1) [right=of w_0] {$w_1$};
   \node (w_2) [below=of w_1,yshift=3cm] {$\cdots$};
   \node[state] (w_n) [below=of w_2,yshift=3cm] {$w_n$};
   \node[state] (w_n1) [left=of w_n] {$w_{n+1}$};
   \node[state] (q) [left=of w_n1] {$q_0$};
    \path[->] 
    (q_0) edge  node {arw(init,0,1)} (w_0)
    (w_0) edge  node  {arw($a_1$,0,$m_{\text{init}}(a_1)$)} (w_1)
    (w_1) edge  node  {arw($a_2$,0,$m_{\text{init}}(a_2)$)} (w_2)
    (w_2) edge  node {arw($a_n$,0,$m_{\text{init}}(a_n)$)} (w_n)
    (w_n) edge  node  {arw(ready,0,1)} (w_n1)
    (w_n1) edge  node  {read(ready,1)} (q)
    (q_0) edge  node  {read(init,1)} (w_n1);
\end{tikzpicture}
\end{figure}

This means that by the atomic-read-write of init, exactly one process can get to the state $w_0$ and initializes the memory accordingly to the specific memory configuration $m_{\text{init}}$. After the initialization is complete, the process sets ready to 1. Now, all processes can reach the state $q_0$. Thus, the control state reachability is preserved.