% !TeX root = root.tex
% !TeX spellcheck = en_US
\section{Vector Addition System with States}

w.l.o.g.
Let us encode the marked Petri Net as
\[N=(S,T,W,M_0)\]
and let us encode the VASS as
\[V=(Q,n,\delta,i_0)\]

\subsection{Task 1}
Given a Petri net $N$ we can encode an equivalent Vass $V$ as:
\[V=(\{1\},|S|,\delta,i_0)\]
and w.o.l.g. assume $S=\{1,\dots,m\}$ let
\[\delta=\{(1,v_t,1)|\forall i \in S : v[i]=W(t,i)-W(i,t)\}\]
i.e. for every transition $t$ in $T$ add a loop to the VASS that encodes this $t$. Lastly let
\[i_0=(1,(M_0[1],\dots,M_0[m]))\]

Given a VASS $V$, w.o.l.g $Q=\{1,\dots,l\}$ we can encode an equivalent Petri net $N$ as:
\[ N=(\{1_V,\dots,n_V \} \cup \{1_Q,\dots,l_Q\},\delta,W,M_0)\]
\[\forall d=(q,v,q') \in \delta, \forall i \in \{1_V,\dots,n_V\}\] 
\[W(q,d)=1, W(d,q)=1\]
\[W(i,d)=
	\begin{cases} 
      -v[i] & v[i]<0\\
      0 & otherwise
	\end{cases}\]
\[W(d,i)=
	\begin{cases} 
      0 & v[i]<0\\
      v[i] & otherwise
	\end{cases}\]
Lastly w.o.l.g $i_0=(x,(y_1,\dots,y_n))$
\[M_0[x]=1\]
\[M_0[i_V]=y_i\]

\subsection{Task 2}

Let $V$ be obtained from $N$ by the above construction. Then a marking $M$ can be transformed to a state for $V$ as follows:
\[\alpha(M)=(1,(M[1],\dots,M[m]))\]

Assume we are in some marking $M$ and transition $t$ is enabled, such that $M[t>M'$, then for $\alpha(M)$ transition $(1,v_t,1)$ from the construction is usable, because markings are positive and our construction encodes the change in markings, thus $\alpha(M)\rightarrow \alpha(M')$. This is sufficient for an argument by induction.

Assume we are in some marking $M$ and transition $t$ is not enabled, then this marking is not enabled, because one state $s\in S$ in $N$ does not have enough tokens for the transition to fire. However this means $M[s]-W(s,t)<0$ and since $v_t[s]=-W(s,t)$ $\alpha(M)$ cannot fire transition $(1,v_t,1)$. Because we have not added any elements to $\delta$ other than elements from $t\ in T$ this is sufficient for an argument by induction.

Let $N$ be obtained from $V$ by the above construction. Then a state for $V$ can be transformed to a marking $M$ as follows:
\[\beta((x,(y_1,\dots,y_n)))=M\]
\[M[i]=
	\begin{cases} 
      1 & i=x\\
      y_j & i=j_V\\
      0 & otherwise
	\end{cases}\]
	
Assume from state $i=(x,(y_1,\dots,y_n))$ we can reach $i'=(x',(y'_1,\dots,y'_n))$ with transition $d=(x,v,x') \in \delta$. Since $(y_1,\dots,y_n)+v=(y'_1,\dots,y'_n) \geq 0$ the equivalently named transition $d$ in $N$ is enabled and $\beta(i)[d>\beta(i')$. This is sufficient for an argument by induction.

Assume from state $i=(x,(y_1,\dots,y_n))$ transition $d=(x,v,x') \in \delta$ cannot be used. Then this is because $\exists j:y_j+v[j]<0$, however this would be $\beta(i)[j]<W(j,d)$ and thus $d$ is not enabled. This is sufficient for an argument by induction.