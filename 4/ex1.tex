% !TeX root = root.tex
% !TeX spellcheck = en_US
\section{Boundedness for transfer and reset nets}
\subsection{Task 1}
Consider the following counter example for transfer nets:
\begin{center}\begin{tikzpicture}[node distance=2 cm,>=stealth',bend angle=45,auto]
	\node[place,tokens=1] (1) [label=below:1]{};
	\node [transition] (t1) [above right of=1,label=above:t1]{}
		edge[pre, bend right = 20] (1)
		edge[post,double,bend left=20] (1);
	\node[place] (2) [below right of=t1,label=below:2]{}
		edge[pre, bend right = 20] node[midway, above]{2} (t1)
		edge[post,double,bend left=20] (t1);
	\node [transition] (t2) [below right of=1,label=above:t2]{}
		edge[post] (1)
		edge[pre] (2);
\end{tikzpicture}\end{center}
For $M = (1,0)$ and $M^\prime = (1,1)$ we have $M < M^\prime$ and $M [t_1 t_2 \rangle M^\prime$, but
\[
M [(t_1 t_2)^2 \rangle (2,1) \neq (1,2) = M + 2 \cdot  (M^\prime - M).
\]

Similarly, we obtain the following counter example for reset nets:
\begin{center}\begin{tikzpicture}[node distance=2 cm,>=stealth',bend angle=45,auto]
	\node[place,tokens=1] (1) [label=below:1]{};
	\node [transition] (t1) [above right of=1,label=above:t1]{}
		edge[pre, bend right = 20] (1);
	\node[place] (2) [below right of=t1,label=below:2]{}
		edge[pre, bend right = 20] node[midway, above]{2} (t1)
		edge[post,double,bend left=20] (t1);
	\node [transition] (t2) [below right of=1,label=above:t2]{}
		edge[post] (1)
		edge[pre] (2);
\end{tikzpicture}\end{center}
For $M = (1,0)$ and $M^\prime = (1,1)$ we have $M < M^\prime$ and $M [t_1 t_2 \rangle M^\prime$, but
\[
M [(t_1 t_2)^2 \rangle (1,1) \neq (1,2) = M + 2 \cdot  (M^\prime - M).
\]

\subsection{Task 2}
The boundedness check for transfer nets is still correct. If we can reach a marking $M^\prime$ from a marking $M$ such that $M < M^\prime$, then we are able to produce arbitrarily many tokens since by transfer edges tokens do not get lost.

The boundedness check for reset nets is not correct anymore. In the example for reset nets above we reach a marking $M^\prime$ from a marking $M$ such that $M < M^\prime$. Thus, our algorithm would return ``unbounded'', but further repetitions of $t_1 t_2$ always result in the marking (1,1) and therefore this sequence does not lead to an unbounded number of tokens. 