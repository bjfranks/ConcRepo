% !TeX root = root.tex
% !TeX spellcheck = en_US
\section{From $\nu$-free Process Algebra to Petri Net}
\subsection{Task 1}
Let $(\nu \vec{a}) \prod_i A_i(\vec{a_i})$ with $\vec{a_i} \subseteq \vec{a}$ be an initial process in $\nu$-free calculus, i.e. no process definition contains restrictions and $\vec{a}$ contains all names.
Because of the condition $\vec{c} \cap \operatorname{bn}(P) = \emptyset$ in the receive rule, we can assume that there are no names in input/output prefixes since all names are bound.

W.l.o.g. we assume that every process definition has one of the following forms:
\begin{enumerate}[(I)]
\item $P(\vec{p}) \stackrel{\text{def}}{=} \pi.Q(\vec{q})$
\item $P(\vec{p}) \stackrel{\text{def}}{=} Q_1(\vec{q_1}) + Q_2(\vec{q_2})$
\item $P(\vec{p}) \stackrel{\text{def}}{=} Q_1(\vec{q_1}) \mid Q_2(\vec{q_2})$
\item $P(\vec{p}) \stackrel{\text{def}}{=} 0$
\end{enumerate}
where $\vec{q},\vec{q_1},\vec{q_2} \subseteq \vec{p}$ and $\pi \in \{!a(), ?a()\}$ for $a \in \vec{p}$.

We construct the corresponding Petri net $N = (S, T, W, M_0)$ as follows:

Set $S$ to the set of all process definitions. Then for each process definition $P(\vec{p}) \in S$ create the following transitions:
\begin{itemize}
\item if $P(\vec{p}) \stackrel{\text{def}}{=} !a().Q(\vec{q})$ create for each $P^\prime(\vec{p^\prime}) \stackrel{\text{def}}{=} ?a().Q^\prime(\vec{q^\prime}) \in S$:
\item[] \begin{center}\begin{tikzpicture}[node distance=2 cm,>=stealth',bend angle=45,auto]
	\node[place] (1) [label=above:$P(\vec{p})$]{};
	\node [transition] (t) [below right of=1]{}
		edge[pre] (1);
	\node[place] (2) [below left of=t,label=below:$P^\prime(\vec{p^\prime})$]{}
		edge[post] (t);
	\node[place] (3) [above right of=t,label=above:$Q(\vec{q})$]{}
		edge[pre] (t);
	\node[place] (4) [below right of=t,label=below:$Q^\prime(\vec{q^\prime})$]{}
		edge[pre] (t);
\end{tikzpicture}\end{center}
\item if $P(\vec{p}) \in S$ is of form (II):
\item[] \begin{center}\begin{tikzpicture}[node distance=2 cm,>=stealth',bend angle=45,auto]
	\node[place] (1) [label=above:$P(\vec{p})$]{};
	\node [transition] (t1) [above right of=1]{}
		edge[pre] (1);
	\node[place] (2) [right of=t1,label=above:$Q_1(\vec{q_1})$]{}
		edge[pre] (t1);
	\node [transition] (t2) [below right of=1]{}
		edge[pre] (1);
	\node[place] (3) [right of=t2,label=below:$Q_2(\vec{q_2})$]{}
		edge[pre] (t2);
\end{tikzpicture}\end{center}
\item if $P(\vec{p}) \in S$ is of form (III):
\item[] \begin{center}\begin{tikzpicture}[node distance=2 cm,>=stealth',bend angle=45,auto]
	\node[place] (1) [label=above:$P(\vec{p})$]{};
	\node [transition] (t) [right of=1]{}
		edge[pre] (1);
	\node[place] (2) [above right of=t,label=above:$Q_1(\vec{q_1})$]{}
		edge[pre] (t);
	\node[place] (3) [below right of=t,label=below:$Q_2(\vec{q_2})$]{}
		edge[pre] (t);
\end{tikzpicture}\end{center}
\item if $P(\vec{p}) \in S$ is of form (IV):
\item[] \begin{center}\begin{tikzpicture}[node distance=2 cm,>=stealth',bend angle=45,auto]
	\node[place] (1) [label=above:$P(\vec{p})$]{};
	\node [transition] (t) [right of=1]{}
		edge[pre] (1);
\end{tikzpicture}\end{center}
\end{itemize}
We set the initial marking $M_0$ such that for each $P(\vec{p}) \in S$ it holds that $M_0(P(\vec{p}))$ is the number of occurrences of $P(\vec{p})$ in the initial process $(\nu \vec{a}) \prod_i A_i(\vec{a_i})$.

By construction the reduction preserves the covering problem.

\subsection{Task 2}
The reduction is not possible with our definition of transfer nets. We need to be able to transfer the tokens of multiple places in one step.