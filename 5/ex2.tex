% !TeX root = root.tex
% !TeX spellcheck = en_US
\section{Some more WQOs}

\subsection{Example 1}

Let $x,y \in X$ with $x < y$, which supposes $|X| \leq 2$, then the sequence $(y),(xy),(xxy),(xxxy), \dots$ is a counterexample. $\leq_{lex}$ is not a WQO.

\subsection{Example 2}

Let $x,y \in X$ with $x < y$, which supposes $|X| \leq 2$, then the sequence $(yy),(xyxy),(xxyxxy),(xxxyxxxy),\dots$ is a counterexample. The argument is basically the y have to be matched by y, otherwise in that position the word is decreasing, and by construction this is not possible. $\leq_{sub}$ is not a WQO.

\subsection{Example 3}

$\leq_{emk}$ is not transitive, by the example $12 \leq_{emk} 0120$ and $0120 \leq_{emk} 01020$, however $12 \not\leq_{emk} 01020$. You dont have to continue reading unless you want to see why it fulfills the sequence criterion.

By Dicksons lemma $(X^i,\leq^i)$ is WQO. We will prove $(\bigcup_{i=1}^k X^i,\bigcup_{i=1}^k \leq^i)$ is well quasi order, this implies by Higman's lemma $\leq_{emk}$ is WQO.

$\bigcup_{i=1}^k \leq^i$ is supposed to mean \[x=(x_1,\dots,x_n) \leq (y_1,\dots,y_m)=y~\textit{iff}~n=m \wedge x \leq^n y\]

I.e. it only compares same sized elements and then uses the $\leq^k$ from the lecture. 

\textit{Proof:} Assume the above mentioned is not WQO, then there exists a sequence of non-increasing elements of $\bigcup_{i=1}^k X^i$. The subsequences of all elements of the same size must be infinite for some $j$. i.e. \[\exists j \in [1,\dots,k]:(s_i)_{\{i|s_i\in X^j\}}~\textit{infinite}\] However this is a contradiction to $(X^j,\leq^j)$ being a WQO. This concludes the proof.

$\leq_{emk}$ is WQO, if $\leq_{emk}$ is QO.

\subsection{Example 4}

By Dickson's lemma the Ordering $(X \times \mathbb{N},\leq^2)$ is a WQO. 

\textit{Proof:} Since $(X,\leq)$ WQO and $(\mathbb{N},\leq)$ WQO. It follows that $(X \cup \mathbb{N},\leq)$ is WQO, where $X$ and $\mathbb{N}$ are uncomparable. Thus by Dickson's lemma $((X \cup \mathbb{N})^2,\leq^2)$ is WQO. We then take the subset $(X \times \mathbb{N},\leq^2)$ and have proven the above. Of course a subset of an order must not be an order, but in our case reflexivity and transitivity is trivial, we were only interested in showing the sequence criterion.

Now we can reformulate a multiset $s$ to be all possible arrangements of sequences of their elements of the form $(a^{m(a)})_{a \in s}$, where $a^{m(a)}$ refers to the tuple $(a,m(a))$ using the definition above(this is for ease of notation). Thus two elements are $\leq_{\mathbb{M}}$ if there exists an arrangement for both sequences such that the sequences are embeddings by highman's lemma. Effectively this holds because our $\leq$ is a weaker version of the embedding $\leq$.

This concludes $\leq_\mathbb{M}$ is WQO