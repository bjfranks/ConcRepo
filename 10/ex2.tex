% !TeX root = root.tex
% !TeX spellcheck = en_US
\section{From $\nu$-free Process Algebra to Petri Net}
\subsection{Task 1}
Let $(\nu \vec{a}) \prod_i A_i(\vec{a_i})$ with $\vec{a_i} \subseteq \vec{a}$ be an initial process in $\nu$-free calculus, i.e. no process definition contains restrictions and $\vec{a}$ contains all possible names.

W.l.o.g. we assume that every process definition has one of the following forms:
\begin{itemize}
\item $P(\vec{p}) \stackrel{\text{def}}{=} !p(\vec{p_1}).Q(\vec{p_2})$
\item $P(\vec{p}) \stackrel{\text{def}}{=} ?p(\vec{x}).Q(\vec{p_1},\vec{x_1})$
\item $P(\vec{p}) \stackrel{\text{def}}{=} Q_1(\vec{p_1}) + Q_2(\vec{p_2})$
\item $P(\vec{p}) \stackrel{\text{def}}{=} Q_1(\vec{p_1}) \mid Q_2(\vec{p_2})$
\item $P(\vec{p}) \stackrel{\text{def}}{=} 0$
\end{itemize}
where $p \in \vec{p}$ and $\vec{p_1},\vec{p_2} \subseteq \vec{p}$ and $\vec{x_1} \subseteq \vec{x}$. So, every process definition contains only one action or control flow element.

We construct the corresponding Petri net $N = (S, T, W, M_0)$ as follows:

Set 
\[S = \{P(\vec{b}) \mid P \text{ process definition}, \vec{b} \subseteq \vec{a} \text{ of suitable length}\}\]
to the set of all uniquely named processes. 

Then create the following transitions:
\begin{itemize}
\item if $P(\vec{b}) = !a(\vec{b_1}).Q(\vec{b_2})$ and
$P^\prime(\vec{b^\prime}) = ?a(\vec{x}).Q^\prime(\vec{b_1^\prime},\vec{x_1})$: 
\item[] \begin{center}\begin{tikzpicture}[node distance=2 cm,>=stealth',bend angle=45,auto]
	\node[place] (1) [label=above:$P(\vec{b})$]{};
	\node [transition] (t) [below right of=1]{}
		edge[pre] (1);
	\node[place] (2) [below left of=t,label=below:$P^\prime(\vec{b^\prime})$]{}
		edge[post] (t);
	\node[place] (3) [above right of=t,label=above:$Q(\vec{b_2})$]{}
		edge[pre] (t);
	\node[place] (4) [below right of=t,label=below:$Q^\prime(\vec{b_2^\prime}{,}\vec{c})$]{}
		edge[pre] (t);
\end{tikzpicture}\end{center}
where $\vec{c} \subseteq \vec{b_1}$ corresponds to $\vec{x_1} \subseteq \vec{x}$ when we substitute $\vec{x}$ by $\vec{b_1}$
\item if $P(\vec{b}) = Q_1(\vec{b_1}) + Q_2(\vec{b_2})$:
\item[] \begin{center}\begin{tikzpicture}[node distance=2 cm,>=stealth',bend angle=45,auto]
	\node[place] (1) [label=above:$P(\vec{b})$]{};
	\node [transition] (t1) [above right of=1]{}
		edge[pre] (1);
	\node[place] (2) [right of=t1,label=above:$Q_1(\vec{b_1})$]{}
		edge[pre] (t1);
	\node [transition] (t2) [below right of=1]{}
		edge[pre] (1);
	\node[place] (3) [right of=t2,label=below:$Q_2(\vec{b_2})$]{}
		edge[pre] (t2);
\end{tikzpicture}\end{center}
\item if $P(\vec{b}) = Q_1(\vec{b_1}) \mid Q_2(\vec{b_2})$:
\item[] \begin{center}\begin{tikzpicture}[node distance=2 cm,>=stealth',bend angle=45,auto]
	\node[place] (1) [label=above:$P(\vec{b})$]{};
	\node [transition] (t) [right of=1]{}
		edge[pre] (1);
	\node[place] (2) [above right of=t,label=above:$Q_1(\vec{b_1})$]{}
		edge[pre] (t);
	\node[place] (3) [below right of=t,label=below:$Q_2(\vec{b_2})$]{}
		edge[pre] (t);
\end{tikzpicture}\end{center}
\item if $P(\vec{b}) = 0$:
\item[] \begin{center}\begin{tikzpicture}[node distance=2 cm,>=stealth',bend angle=45,auto]
	\node[place] (1) [label=above:$P(\vec{b})$]{};
	\node [transition] (t) [right of=1]{}
		edge[pre] (1);
\end{tikzpicture}\end{center}
\end{itemize}
We set the initial marking $M_0$ such that for each $P(\vec{b}) \in S$ it holds that $M_0(P(\vec{b}))$ is the number of occurrences of $P(\vec{b})$ in the initial process $(\nu \vec{a}) \prod_i A_i(\vec{a_i})$.

By construction the reduction preserves the covering problem.

\subsection{Task 2}
The reduction is not possible with our definition of transfer nets. We need to be able to transfer the tokens of multiple places in one step since there can be multiple different kinds of processes that receive the broadcast message in one step.