% !TeX root = root.tex
% !TeX spellcheck = en_US
\section{Weak Bisimulation}
\begin{align*}
& (\nu c)((?a.!c.0|?b.!c.0)&|?c.?c.P) \\
\equiv & (\nu c)(?a.(!c.0|?b.!c.0+?b.(?a.!c.0|!c.0))&|?c.?c.P)
\end{align*}
\begin{align*}
& (\nu c)(?a.(!c.0|?b.!c.0+?b.(?a.!c.0|!c.0))|?c.?c.P)&&&&\\
\equiv & (\nu c)(?a.(!c.0|?b.!c.0)|?c.?c.P&+& ?b.(?a.!c.0|!c.0))|?c.?c.P &+& ?c.R)\\
\equiv & (\nu c)(?a.(!c.0|?b.!c.0)|?c.?c.P)&+& (\nu c)(?b.(?a.!c.0|!c.0))|?c.?c.P) &+& (\nu c)(?c.R)\\
\equiv & (\nu c)(?a.(!c.0|?b.!c.0)|?c.?c.P)&+& (\nu c)(?b.(?a.!c.0|!c.0))|?c.?c.P) &+& 0\\
\equiv & (\nu c)(?a.(!c.0|?b.!c.0)|?c.?c.P)&+& (\nu c)(?b.(?a.!c.0|!c.0))|?c.?c.P) &&\\
\end{align*}

With some analog steps we reach

\begin{align*}
?a.(\nu c)((!c.0|?b.!c.0)|?c.?c.P)+?b.(\nu c)((?a.!c.0|!c.0))|?c.?c.P)
\end{align*}

We only consider one of the $+$ and ignore the $?a.$

\begin{align*}
&(\nu c)((!c.0|?b.!c.0)|?c.?c.P)\\
\equiv &(\nu c)((!c.(0|?b.!c.0)+?b.(!c.0|!c.0))|?c.?c.P)\\
\equiv &(\nu c)( ?b.(!c.0|!c.0|?c.?c.P) + (0|?b.!c.0|?c.P))\\
\equiv &?b.((\nu c)(!c.0|!c.0|?c.?c.P)) + (\nu c)(0|?b.!c.0|?c.P)
\end{align*}

With multiple steps ignoring resulting 0s we get from the left $+$, ignoring the $?b.$
\begin{align*}
&(\nu c)(!c.0|!c.0|?c.?c.P)\\
\equiv &(\nu c)(0|0|P)\\
\equiv &(\nu c) P\\
\equiv &P
\end{align*}

And the right $+$

\begin{align*}
&(\nu c)(0|?b.!c.0|?c.P)\\
\equiv &(\nu c)(?b.!c.0|?c.P)\\
\equiv &(\nu c)(?b.(!c.0|?c.P))\\
\equiv &?b.((\nu c)(!c.0|?c.P))\\
\equiv &?b.((\nu c)(0|P))\\
\equiv &?b.((\nu c) P)\\
\equiv &?b.P
\end{align*}

Both are equal and thus we finally get that the initial left side is equivalent to (we ignored the $?a.$ and the $?b.$ in the first of the two)

\begin{align*}
&?a.(\nu c)((!c.0|?b.!c.0)|?c.?c.P)\\
\equiv &?a.?b.P
\end{align*}

We can do the same analog for the right side and thus get

\begin{align*}
&(\nu c)((?a.!c.0|?b.!c.0)|?c.?c.P)\\
\equiv &?a.?b.P + ?b.?a.P
\end{align*}

Which is what was asked to prove.
